% Options for packages loaded elsewhere
\PassOptionsToPackage{unicode}{hyperref}
\PassOptionsToPackage{hyphens}{url}
%
\documentclass[
]{article}
\usepackage{lmodern}
\usepackage{amssymb,amsmath}
\usepackage{ifxetex,ifluatex}
\ifnum 0\ifxetex 1\fi\ifluatex 1\fi=0 % if pdftex
  \usepackage[T1]{fontenc}
  \usepackage[utf8]{inputenc}
  \usepackage{textcomp} % provide euro and other symbols
\else % if luatex or xetex
  \usepackage{unicode-math}
  \defaultfontfeatures{Scale=MatchLowercase}
  \defaultfontfeatures[\rmfamily]{Ligatures=TeX,Scale=1}
\fi
% Use upquote if available, for straight quotes in verbatim environments
\IfFileExists{upquote.sty}{\usepackage{upquote}}{}
\IfFileExists{microtype.sty}{% use microtype if available
  \usepackage[]{microtype}
  \UseMicrotypeSet[protrusion]{basicmath} % disable protrusion for tt fonts
}{}
\makeatletter
\@ifundefined{KOMAClassName}{% if non-KOMA class
  \IfFileExists{parskip.sty}{%
    \usepackage{parskip}
  }{% else
    \setlength{\parindent}{0pt}
    \setlength{\parskip}{6pt plus 2pt minus 1pt}}
}{% if KOMA class
  \KOMAoptions{parskip=half}}
\makeatother
\usepackage{xcolor}
\IfFileExists{xurl.sty}{\usepackage{xurl}}{} % add URL line breaks if available
\IfFileExists{bookmark.sty}{\usepackage{bookmark}}{\usepackage{hyperref}}
\hypersetup{
  pdftitle={First Project},
  pdfauthor={20180968 박서현},
  hidelinks,
  pdfcreator={LaTeX via pandoc}}
\urlstyle{same} % disable monospaced font for URLs
\usepackage[margin=1in]{geometry}
\usepackage{color}
\usepackage{fancyvrb}
\newcommand{\VerbBar}{|}
\newcommand{\VERB}{\Verb[commandchars=\\\{\}]}
\DefineVerbatimEnvironment{Highlighting}{Verbatim}{commandchars=\\\{\}}
% Add ',fontsize=\small' for more characters per line
\usepackage{framed}
\definecolor{shadecolor}{RGB}{248,248,248}
\newenvironment{Shaded}{\begin{snugshade}}{\end{snugshade}}
\newcommand{\AlertTok}[1]{\textcolor[rgb]{0.94,0.16,0.16}{#1}}
\newcommand{\AnnotationTok}[1]{\textcolor[rgb]{0.56,0.35,0.01}{\textbf{\textit{#1}}}}
\newcommand{\AttributeTok}[1]{\textcolor[rgb]{0.77,0.63,0.00}{#1}}
\newcommand{\BaseNTok}[1]{\textcolor[rgb]{0.00,0.00,0.81}{#1}}
\newcommand{\BuiltInTok}[1]{#1}
\newcommand{\CharTok}[1]{\textcolor[rgb]{0.31,0.60,0.02}{#1}}
\newcommand{\CommentTok}[1]{\textcolor[rgb]{0.56,0.35,0.01}{\textit{#1}}}
\newcommand{\CommentVarTok}[1]{\textcolor[rgb]{0.56,0.35,0.01}{\textbf{\textit{#1}}}}
\newcommand{\ConstantTok}[1]{\textcolor[rgb]{0.00,0.00,0.00}{#1}}
\newcommand{\ControlFlowTok}[1]{\textcolor[rgb]{0.13,0.29,0.53}{\textbf{#1}}}
\newcommand{\DataTypeTok}[1]{\textcolor[rgb]{0.13,0.29,0.53}{#1}}
\newcommand{\DecValTok}[1]{\textcolor[rgb]{0.00,0.00,0.81}{#1}}
\newcommand{\DocumentationTok}[1]{\textcolor[rgb]{0.56,0.35,0.01}{\textbf{\textit{#1}}}}
\newcommand{\ErrorTok}[1]{\textcolor[rgb]{0.64,0.00,0.00}{\textbf{#1}}}
\newcommand{\ExtensionTok}[1]{#1}
\newcommand{\FloatTok}[1]{\textcolor[rgb]{0.00,0.00,0.81}{#1}}
\newcommand{\FunctionTok}[1]{\textcolor[rgb]{0.00,0.00,0.00}{#1}}
\newcommand{\ImportTok}[1]{#1}
\newcommand{\InformationTok}[1]{\textcolor[rgb]{0.56,0.35,0.01}{\textbf{\textit{#1}}}}
\newcommand{\KeywordTok}[1]{\textcolor[rgb]{0.13,0.29,0.53}{\textbf{#1}}}
\newcommand{\NormalTok}[1]{#1}
\newcommand{\OperatorTok}[1]{\textcolor[rgb]{0.81,0.36,0.00}{\textbf{#1}}}
\newcommand{\OtherTok}[1]{\textcolor[rgb]{0.56,0.35,0.01}{#1}}
\newcommand{\PreprocessorTok}[1]{\textcolor[rgb]{0.56,0.35,0.01}{\textit{#1}}}
\newcommand{\RegionMarkerTok}[1]{#1}
\newcommand{\SpecialCharTok}[1]{\textcolor[rgb]{0.00,0.00,0.00}{#1}}
\newcommand{\SpecialStringTok}[1]{\textcolor[rgb]{0.31,0.60,0.02}{#1}}
\newcommand{\StringTok}[1]{\textcolor[rgb]{0.31,0.60,0.02}{#1}}
\newcommand{\VariableTok}[1]{\textcolor[rgb]{0.00,0.00,0.00}{#1}}
\newcommand{\VerbatimStringTok}[1]{\textcolor[rgb]{0.31,0.60,0.02}{#1}}
\newcommand{\WarningTok}[1]{\textcolor[rgb]{0.56,0.35,0.01}{\textbf{\textit{#1}}}}
\usepackage{graphicx,grffile}
\makeatletter
\def\maxwidth{\ifdim\Gin@nat@width>\linewidth\linewidth\else\Gin@nat@width\fi}
\def\maxheight{\ifdim\Gin@nat@height>\textheight\textheight\else\Gin@nat@height\fi}
\makeatother
% Scale images if necessary, so that they will not overflow the page
% margins by default, and it is still possible to overwrite the defaults
% using explicit options in \includegraphics[width, height, ...]{}
\setkeys{Gin}{width=\maxwidth,height=\maxheight,keepaspectratio}
% Set default figure placement to htbp
\makeatletter
\def\fps@figure{htbp}
\makeatother
\setlength{\emergencystretch}{3em} % prevent overfull lines
\providecommand{\tightlist}{%
  \setlength{\itemsep}{0pt}\setlength{\parskip}{0pt}}
\setcounter{secnumdepth}{-\maxdimen} % remove section numbering

\title{First Project}
\author{20180968 박서현}
\date{2020 10 11}

\begin{document}
\maketitle

\hypertarget{uxb370uxc774uxd130-uxbd88uxb7ecuxc624uxae30}{%
\subsection{데이터
불러오기}\label{uxb370uxc774uxd130-uxbd88uxb7ecuxc624uxae30}}

\begin{Shaded}
\begin{Highlighting}[]
\KeywordTok{library}\NormalTok{(tidyverse)}
\end{Highlighting}
\end{Shaded}

\begin{verbatim}
## -- Attaching packages --------------------------------------------------- tidyverse 1.3.0 --
\end{verbatim}

\begin{verbatim}
## v ggplot2 3.3.2     v purrr   0.3.4
## v tibble  3.0.3     v dplyr   1.0.2
## v tidyr   1.1.2     v stringr 1.4.0
## v readr   1.4.0     v forcats 0.5.0
\end{verbatim}

\begin{verbatim}
## -- Conflicts ------------------------------------------------------ tidyverse_conflicts() --
## x dplyr::filter() masks stats::filter()
## x dplyr::lag()    masks stats::lag()
\end{verbatim}

\begin{Shaded}
\begin{Highlighting}[]
\NormalTok{DF <-}\StringTok{ }\KeywordTok{read_csv}\NormalTok{(}\StringTok{"C:/Users/nabib/Documents/GitHub/Statics/1011/toefl.csv"}\NormalTok{)}
\end{Highlighting}
\end{Shaded}

\begin{verbatim}
## 
## -- Column specification --------------------------------------------------------------------
## cols(
##   id = col_double(),
##   gender = col_character(),
##   listen = col_double(),
##   grammar = col_double(),
##   reading = col_double()
## )
\end{verbatim}

\begin{Shaded}
\begin{Highlighting}[]
\KeywordTok{head}\NormalTok{(DF)}
\end{Highlighting}
\end{Shaded}

\begin{verbatim}
## # A tibble: 6 x 5
##      id gender listen grammar reading
##   <dbl> <chr>   <dbl>   <dbl>   <dbl>
## 1     1 FEMALE     40      50      56
## 2     2 FEMALE     33      44      46
## 3     3 MALE       39      44      47
## 4     4 FEMALE     44      55      66
## 5     5 MALE       43      56      53
## 6     6 MALE       57      49      53
\end{verbatim}

\hypertarget{uxc131femalemaleuxb97c-uxc0c8uxb85cuxc6b4-uxbcc0uxc218-uxc131uxbcc40uxb0a81uxc5ecuxb85c-uxbcc0uxd658uxd558uxace0-uxac12uxc744-uxc9c0uxc815uxd558uxc2dcuxc624.}{%
\subsection{성(Female,Male\}를 새로운 변수 성별\{0=남,1=여\}로 변환하고
값을
지정하시오.}\label{uxc131femalemaleuxb97c-uxc0c8uxb85cuxc6b4-uxbcc0uxc218-uxc131uxbcc40uxb0a81uxc5ecuxb85c-uxbcc0uxd658uxd558uxace0-uxac12uxc744-uxc9c0uxc815uxd558uxc2dcuxc624.}}

변수유형 변경(factor)

\begin{Shaded}
\begin{Highlighting}[]
\NormalTok{DF <-}\StringTok{ }\NormalTok{DF }\OperatorTok\StringTok{ }\KeywordTok{mutate}\NormalTok{(}\DataTypeTok{gender=}\KeywordTok{factor}\NormalTok{(gender))}
\KeywordTok{str}\NormalTok{(DF)}
\end{Highlighting}
\end{Shaded}

\begin{verbatim}
## tibble [100 x 5] (S3: spec_tbl_df/tbl_df/tbl/data.frame)
##  $ id     : num [1:100] 1 2 3 4 5 6 7 8 9 10 ...
##  $ gender : Factor w/ 2 levels "FEMALE","MALE": 1 1 2 1 2 2 1 1 1 1 ...
##  $ listen : num [1:100] 40 33 39 44 43 57 54 40 43 50 ...
##  $ grammar: num [1:100] 50 44 44 55 56 49 43 50 51 48 ...
##  $ reading: num [1:100] 56 46 47 66 53 53 56 53 57 52 ...
##  - attr(*, "spec")=
##   .. cols(
##   ..   id = col_double(),
##   ..   gender = col_character(),
##   ..   listen = col_double(),
##   ..   grammar = col_double(),
##   ..   reading = col_double()
##   .. )
\end{verbatim}

수준재조정

\begin{Shaded}
\begin{Highlighting}[]
\NormalTok{DF}\OperatorTok{$}\NormalTok{gender <-}\StringTok{ }\KeywordTok{factor}\NormalTok{(DF}\OperatorTok{$}\NormalTok{gender, }\DataTypeTok{levels=}\KeywordTok{c}\NormalTok{(}\StringTok{'MALE'}\NormalTok{, }\StringTok{'FEMALE'}\NormalTok{))}

\NormalTok{DF}\OperatorTok{$}\NormalTok{gender}
\end{Highlighting}
\end{Shaded}

\begin{verbatim}
##   [1] FEMALE FEMALE MALE   FEMALE MALE   MALE   FEMALE FEMALE FEMALE FEMALE
##  [11] MALE   MALE   MALE   MALE   FEMALE FEMALE FEMALE MALE   FEMALE FEMALE
##  [21] FEMALE MALE   FEMALE FEMALE MALE   MALE   FEMALE FEMALE MALE   FEMALE
##  [31] MALE   MALE   FEMALE FEMALE MALE   FEMALE MALE   FEMALE FEMALE FEMALE
##  [41] MALE   FEMALE FEMALE MALE   FEMALE MALE   FEMALE MALE   FEMALE MALE  
##  [51] FEMALE FEMALE FEMALE FEMALE MALE   MALE   FEMALE FEMALE MALE   MALE  
##  [61] FEMALE MALE   MALE   FEMALE FEMALE FEMALE MALE   FEMALE MALE   MALE  
##  [71] MALE   FEMALE FEMALE MALE   FEMALE FEMALE FEMALE FEMALE FEMALE FEMALE
##  [81] MALE   FEMALE MALE   FEMALE MALE   FEMALE FEMALE FEMALE FEMALE FEMALE
##  [91] MALE   FEMALE MALE   FEMALE MALE   FEMALE MALE   MALE   FEMALE MALE  
## Levels: MALE FEMALE
\end{verbatim}

factor를 수치 변수로 변환

\begin{Shaded}
\begin{Highlighting}[]
\NormalTok{DF}\OperatorTok{$}\NormalTok{gendernum <-}\KeywordTok{as.numeric}\NormalTok{(DF}\OperatorTok{$}\NormalTok{gender)}

\KeywordTok{head}\NormalTok{(DF)}
\end{Highlighting}
\end{Shaded}

\begin{verbatim}
## # A tibble: 6 x 6
##      id gender listen grammar reading gendernum
##   <dbl> <fct>   <dbl>   <dbl>   <dbl>     <dbl>
## 1     1 FEMALE     40      50      56         2
## 2     2 FEMALE     33      44      46         2
## 3     3 MALE       39      44      47         1
## 4     4 FEMALE     44      55      66         2
## 5     5 MALE       43      56      53         1
## 6     6 MALE       57      49      53         1
\end{verbatim}

값 조정

\begin{Shaded}
\begin{Highlighting}[]
\NormalTok{DF <-}\StringTok{ }\KeywordTok{mutate}\NormalTok{(DF, }\DataTypeTok{gendernum =} \KeywordTok{ifelse}\NormalTok{(gendernum}\OperatorTok{==}\DecValTok{2}\NormalTok{, }\DecValTok{1}\NormalTok{, }\DecValTok{0}\NormalTok{))}

\KeywordTok{head}\NormalTok{(DF)}
\end{Highlighting}
\end{Shaded}

\begin{verbatim}
## # A tibble: 6 x 6
##      id gender listen grammar reading gendernum
##   <dbl> <fct>   <dbl>   <dbl>   <dbl>     <dbl>
## 1     1 FEMALE     40      50      56         1
## 2     2 FEMALE     33      44      46         1
## 3     3 MALE       39      44      47         0
## 4     4 FEMALE     44      55      66         1
## 5     5 MALE       43      56      53         0
## 6     6 MALE       57      49      53         0
\end{verbatim}

\hypertarget{uxd1a0uxd50cuxc131uxc801uxc740-uxb4e3uxae30uxbb38uxbc95uxb3c5uxd574310uxb97c-uxc18cuxc218uxc810-uxccabuxc9f8uxc790uxb9acuxc5d0uxc11c-uxbc18uxc62cuxb9bcuxd55c-uxac83uxc774uxb2e4.-uxc0c8-uxbcc0uxc218-uxc131uxc801uxc744-uxc0dduxc131uxd558uxc2dcuxc624.}{%
\subsection{토플성적은 (듣기+문법+독해)/3*10를 소수점 첫째자리에서
반올림한 것이다. 새 변수 성적을
생성하시오.}\label{uxd1a0uxd50cuxc131uxc801uxc740-uxb4e3uxae30uxbb38uxbc95uxb3c5uxd574310uxb97c-uxc18cuxc218uxc810-uxccabuxc9f8uxc790uxb9acuxc5d0uxc11c-uxbc18uxc62cuxb9bcuxd55c-uxac83uxc774uxb2e4.-uxc0c8-uxbcc0uxc218-uxc131uxc801uxc744-uxc0dduxc131uxd558uxc2dcuxc624.}}

\begin{Shaded}
\begin{Highlighting}[]
\NormalTok{DF <-}\StringTok{ }\KeywordTok{mutate}\NormalTok{(DF, }\DataTypeTok{grade =} \KeywordTok{round}\NormalTok{((listen }\OperatorTok{+}\StringTok{ }\NormalTok{grammar }\OperatorTok{+}\StringTok{ }\NormalTok{reading) }\OperatorTok{/}\StringTok{ }\DecValTok{3} \OperatorTok{*}\StringTok{ }\DecValTok{10}\NormalTok{), }\DecValTok{1}\NormalTok{)}

\KeywordTok{head}\NormalTok{(DF)}
\end{Highlighting}
\end{Shaded}

\begin{verbatim}
## # A tibble: 6 x 8
##      id gender listen grammar reading gendernum grade   `1`
##   <dbl> <fct>   <dbl>   <dbl>   <dbl>     <dbl> <dbl> <dbl>
## 1     1 FEMALE     40      50      56         1   487     1
## 2     2 FEMALE     33      44      46         1   410     1
## 3     3 MALE       39      44      47         0   433     1
## 4     4 FEMALE     44      55      66         1   550     1
## 5     5 MALE       43      56      53         0   507     1
## 6     6 MALE       57      49      53         0   530     1
\end{verbatim}

\hypertarget{uxc131uxc801uxc758-uxd3c9uxade0-uxbd84uxc0b0-uxd45cuxc900uxd3b8uxcc28-uxcd5cuxc18c-uxcd5cuxb300uxac12uxc744-uxacc4uxc0b0uxd558uxc2dcuxc624.-uxc0c1uxc790uxadf8uxb9bc-uxd788uxc2a4uxd1a0uxadf8uxb7a8-uxc815uxaddcuxd655uxb960uxadf8uxb9bcuxc744-uxc791uxc131uxd558uxc2dcuxc624}{%
\subsection{성적의 평균, 분산, 표준편차, 최소, 최대값을 계산하시오.
상자그림, 히스토그램, 정규확률그림을
작성하시오}\label{uxc131uxc801uxc758-uxd3c9uxade0-uxbd84uxc0b0-uxd45cuxc900uxd3b8uxcc28-uxcd5cuxc18c-uxcd5cuxb300uxac12uxc744-uxacc4uxc0b0uxd558uxc2dcuxc624.-uxc0c1uxc790uxadf8uxb9bc-uxd788uxc2a4uxd1a0uxadf8uxb7a8-uxc815uxaddcuxd655uxb960uxadf8uxb9bcuxc744-uxc791uxc131uxd558uxc2dcuxc624}}

성적의 평균, 분산, 표준편차, 최소, 최대값 계산

\begin{Shaded}
\begin{Highlighting}[]
\KeywordTok{summarize}\NormalTok{(DF, }\DataTypeTok{n=}\KeywordTok{n}\NormalTok{(), }\DataTypeTok{mean=}\KeywordTok{mean}\NormalTok{(grade), }\DataTypeTok{var=}\KeywordTok{var}\NormalTok{(grade), }\DataTypeTok{sd=}\KeywordTok{sd}\NormalTok{(grade), }\DataTypeTok{min =} \KeywordTok{min}\NormalTok{(grade), }\KeywordTok{max}\NormalTok{(grade))}
\end{Highlighting}
\end{Shaded}

\begin{verbatim}
## # A tibble: 1 x 6
##       n  mean   var    sd   min `max(grade)`
##   <int> <dbl> <dbl> <dbl> <dbl>        <dbl>
## 1   100  505. 4003.  63.3   350          647
\end{verbatim}

히스토그램

\begin{Shaded}
\begin{Highlighting}[]
\KeywordTok{ggplot}\NormalTok{(DF, }\KeywordTok{aes}\NormalTok{(}\DataTypeTok{x=}\NormalTok{grade)) }\OperatorTok{+}\StringTok{ }\KeywordTok{geom_histogram}\NormalTok{()}
\end{Highlighting}
\end{Shaded}

\begin{verbatim}
## `stat_bin()` using `bins = 30`. Pick better value with `binwidth`.
\end{verbatim}

\includegraphics{First_Project_R_files/figure-latex/unnamed-chunk-9-1.pdf}

상자 그림

\begin{Shaded}
\begin{Highlighting}[]
\KeywordTok{boxplot}\NormalTok{(DF}\OperatorTok{$}\NormalTok{grade)}
\end{Highlighting}
\end{Shaded}

\includegraphics{First_Project_R_files/figure-latex/unnamed-chunk-10-1.pdf}

\hypertarget{uxc131uxbcc4-uxc131uxc801uxc758-uxd3c9uxade0-uxbd84uxc0b0-uxd45cuxc900uxd3b8uxcc28-uxcd5cuxc18c-uxcd5cuxb300uxac12uxc744-uxacc4uxc0b0uxd558uxc2dcuxc624.-uxc131uxbcc4-uxc0c1uxc790uxadf8uxb9bcuxd788uxc2a4uxd1a0uxadf8uxb7a8-uxc815uxaddcuxd655uxb960uxadf8uxb9bcuxb4f1uxc744-uxc791uxc131uxd558uxc2dcuxc624.}{%
\subsection{성별 성적의 평균, 분산, 표준편차, 최소, 최대값을 계산하시오.
성별 상자그림,히스토그램, 정규확률그림등을
작성하시오.}\label{uxc131uxbcc4-uxc131uxc801uxc758-uxd3c9uxade0-uxbd84uxc0b0-uxd45cuxc900uxd3b8uxcc28-uxcd5cuxc18c-uxcd5cuxb300uxac12uxc744-uxacc4uxc0b0uxd558uxc2dcuxc624.-uxc131uxbcc4-uxc0c1uxc790uxadf8uxb9bcuxd788uxc2a4uxd1a0uxadf8uxb7a8-uxc815uxaddcuxd655uxb960uxadf8uxb9bcuxb4f1uxc744-uxc791uxc131uxd558uxc2dcuxc624.}}

성별 성적의 평균, 분산, 표준편차, 최소, 최대값

\begin{Shaded}
\begin{Highlighting}[]
\NormalTok{DF }\OperatorTok
\StringTok{  }\KeywordTok{group_by}\NormalTok{(gender) }\OperatorTok
\StringTok{  }\KeywordTok{summarize_at}\NormalTok{(}\KeywordTok{vars}\NormalTok{(grade), }\KeywordTok{list}\NormalTok{(}\DataTypeTok{mean=}\NormalTok{mean, }\DataTypeTok{var=}\NormalTok{var, }\DataTypeTok{sd=}\NormalTok{sd, }\DataTypeTok{min =}\NormalTok{ min, max))}
\end{Highlighting}
\end{Shaded}

\begin{verbatim}
## # A tibble: 2 x 6
##   gender  mean   var    sd   min   fn1
##   <fct>  <dbl> <dbl> <dbl> <dbl> <dbl>
## 1 MALE    510. 3921.  62.6   350   647
## 2 FEMALE  501. 4097.  64.0   350   640
\end{verbatim}

성별 히스토그램

\begin{Shaded}
\begin{Highlighting}[]
\KeywordTok{ggplot}\NormalTok{(DF) }\OperatorTok{+}
\StringTok{  }\KeywordTok{geom_histogram}\NormalTok{(}\KeywordTok{aes}\NormalTok{(}\DataTypeTok{x=}\NormalTok{grade, }\DataTypeTok{y=}\NormalTok{..density.., }\DataTypeTok{fill=}\NormalTok{gender), }\DataTypeTok{alpha=}\FloatTok{0.5}\NormalTok{) }\OperatorTok{+}
\StringTok{  }\KeywordTok{geom_density}\NormalTok{(}\KeywordTok{aes}\NormalTok{(}\DataTypeTok{x=}\NormalTok{grade, }\DataTypeTok{fill=}\NormalTok{gender), }\DataTypeTok{alpha=}\FloatTok{0.5}\NormalTok{)}
\end{Highlighting}
\end{Shaded}

\begin{verbatim}
## `stat_bin()` using `bins = 30`. Pick better value with `binwidth`.
\end{verbatim}

\includegraphics{First_Project_R_files/figure-latex/unnamed-chunk-12-1.pdf}

성별 상자그림

\begin{Shaded}
\begin{Highlighting}[]
\KeywordTok{boxplot}\NormalTok{(grade}\OperatorTok{~}\NormalTok{gender, }\DataTypeTok{data=}\NormalTok{DF)}
\end{Highlighting}
\end{Shaded}

\includegraphics{First_Project_R_files/figure-latex/unnamed-chunk-13-1.pdf}

\hypertarget{uxb4e3uxae30-uxbb38uxbc95-uxb3c5uxd574-uxc131uxc801uxb4e4uxc758-uxc0b0uxc810uxb3c4uxb97c-uxadf8uxb9acuxc2dcuxc624.}{%
\subsection{듣기, 문법, 독해, 성적들의 산점도를
그리시오.}\label{uxb4e3uxae30-uxbb38uxbc95-uxb3c5uxd574-uxc131uxc801uxb4e4uxc758-uxc0b0uxc810uxb3c4uxb97c-uxadf8uxb9acuxc2dcuxc624.}}

듣기 vs 문법

\begin{Shaded}
\begin{Highlighting}[]
\KeywordTok{plot}\NormalTok{(DF}\OperatorTok{$}\NormalTok{listen, DF}\OperatorTok{$}\NormalTok{grammar, }\DataTypeTok{type=}\StringTok{'n'}\NormalTok{)}
\KeywordTok{text}\NormalTok{(DF}\OperatorTok{$}\NormalTok{listen, DF}\OperatorTok{$}\NormalTok{grammar, }\DataTypeTok{label=}\NormalTok{DF}\OperatorTok{$}\NormalTok{gendernum)}
\end{Highlighting}
\end{Shaded}

\includegraphics{First_Project_R_files/figure-latex/unnamed-chunk-14-1.pdf}

읽기 vs 성적

\begin{Shaded}
\begin{Highlighting}[]
\KeywordTok{plot}\NormalTok{(DF}\OperatorTok{$}\NormalTok{reading, DF}\OperatorTok{$}\NormalTok{grade, }\DataTypeTok{type=}\StringTok{'n'}\NormalTok{)}
\KeywordTok{text}\NormalTok{(DF}\OperatorTok{$}\NormalTok{reading, DF}\OperatorTok{$}\NormalTok{grade, }\DataTypeTok{label=}\NormalTok{DF}\OperatorTok{$}\NormalTok{gendernum)}
\end{Highlighting}
\end{Shaded}

\includegraphics{First_Project_R_files/figure-latex/unnamed-chunk-15-1.pdf}

\hypertarget{uxc131uxc801uxc774-500uxc774uxc0c1uxc774uxba74-1uxc751uxc2dcuxb8ccuxc640-uxc878uxc5c5uxc2dcuxd5d8uxba74uxc81c-450uxc774uxc0c1-500uxbbf8uxb9ccuxc774uxba74-2uxc878uxc5c5uxc2dcuxd5d8uxba74uxc81c-400uxc774uxc0c1-450uxbbf8uxb9ccuxc774uxba74-3uxc9c0uxc6d0uxc5c6uxc74c-400uxbbf8uxb9ccuxc774uxba74-4uxd2b9uxac15uxc218uxac15-uxd558uxae30uxb85c-uxd558uxc600uxb2e4.-uxbcc0uxc218-uxc9c0uxc6d01234uxc744-uxc791uxc131uxd558uxc2dcuxc624}{%
\subsection{성적이 500이상이면 1=응시료와 졸업시험면제, 450이상
500미만이면 2=졸업시험면제, 400이상 450미만이면 3=지원없음, 400미만이면
4=특강수강 하기로 하였다. 변수 지원\{1,2,3,4\}을
작성하시오}\label{uxc131uxc801uxc774-500uxc774uxc0c1uxc774uxba74-1uxc751uxc2dcuxb8ccuxc640-uxc878uxc5c5uxc2dcuxd5d8uxba74uxc81c-450uxc774uxc0c1-500uxbbf8uxb9ccuxc774uxba74-2uxc878uxc5c5uxc2dcuxd5d8uxba74uxc81c-400uxc774uxc0c1-450uxbbf8uxb9ccuxc774uxba74-3uxc9c0uxc6d0uxc5c6uxc74c-400uxbbf8uxb9ccuxc774uxba74-4uxd2b9uxac15uxc218uxac15-uxd558uxae30uxb85c-uxd558uxc600uxb2e4.-uxbcc0uxc218-uxc9c0uxc6d01234uxc744-uxc791uxc131uxd558uxc2dcuxc624}}

\begin{Shaded}
\begin{Highlighting}[]
\NormalTok{DF}\OperatorTok{$}\NormalTok{support <-}\StringTok{ }\KeywordTok{ifelse}\NormalTok{(DF}\OperatorTok{$}\NormalTok{grade }\OperatorTok{>=}\StringTok{ }\DecValTok{500}\NormalTok{, }\DecValTok{1}\NormalTok{,}
                     \KeywordTok{ifelse}\NormalTok{(DF}\OperatorTok{$}\NormalTok{grade }\OperatorTok{>=}\StringTok{ }\DecValTok{450}\NormalTok{, }\DecValTok{2}\NormalTok{, }
                            \KeywordTok{ifelse}\NormalTok{(DF}\OperatorTok{$}\NormalTok{grade }\OperatorTok{>=}\StringTok{ }\DecValTok{400}\NormalTok{, }\DecValTok{3}\NormalTok{, }\DecValTok{4}\NormalTok{)))}

\KeywordTok{head}\NormalTok{(DF)}
\end{Highlighting}
\end{Shaded}

\begin{verbatim}
## # A tibble: 6 x 9
##      id gender listen grammar reading gendernum grade   `1` support
##   <dbl> <fct>   <dbl>   <dbl>   <dbl>     <dbl> <dbl> <dbl>   <dbl>
## 1     1 FEMALE     40      50      56         1   487     1       2
## 2     2 FEMALE     33      44      46         1   410     1       3
## 3     3 MALE       39      44      47         0   433     1       3
## 4     4 FEMALE     44      55      66         1   550     1       1
## 5     5 MALE       43      56      53         0   507     1       1
## 6     6 MALE       57      49      53         0   530     1       1
\end{verbatim}

\hypertarget{uxc131uxc801uxc774-450uxc774uxc0c1uxc778-uxd559uxc0dduxb4e4uxc740-uxc878uxc5c5uxc2dcuxd5d8uxc744-uxba74uxc81cuxd574uxc8fcuxae30uxb85c-uxd558uxc600uxb2e4.-uxbcc0uxc218-uxba74uxc81c-1uxc131uxacf50uxc2e4uxd328uxb97c-uxb9ccuxb4dcuxc2dcuxc624.}{%
\subsection{성적이 450이상인 학생들은 졸업시험을 면제해주기로 하였다.
변수 면제 (1=성공,0=실패)를
만드시오.}\label{uxc131uxc801uxc774-450uxc774uxc0c1uxc778-uxd559uxc0dduxb4e4uxc740-uxc878uxc5c5uxc2dcuxd5d8uxc744-uxba74uxc81cuxd574uxc8fcuxae30uxb85c-uxd558uxc600uxb2e4.-uxbcc0uxc218-uxba74uxc81c-1uxc131uxacf50uxc2e4uxd328uxb97c-uxb9ccuxb4dcuxc2dcuxc624.}}

\begin{Shaded}
\begin{Highlighting}[]
\NormalTok{DF}\OperatorTok{$}\NormalTok{exemption <-}\StringTok{ }\KeywordTok{ifelse}\NormalTok{(DF}\OperatorTok{$}\NormalTok{support }\OperatorTok{<=}\StringTok{ }\DecValTok{2}\NormalTok{, }\DecValTok{1}\NormalTok{, }\DecValTok{0}\NormalTok{)}

\KeywordTok{head}\NormalTok{(DF)}
\end{Highlighting}
\end{Shaded}

\begin{verbatim}
## # A tibble: 6 x 10
##      id gender listen grammar reading gendernum grade   `1` support exemption
##   <dbl> <fct>   <dbl>   <dbl>   <dbl>     <dbl> <dbl> <dbl>   <dbl>     <dbl>
## 1     1 FEMALE     40      50      56         1   487     1       2         1
## 2     2 FEMALE     33      44      46         1   410     1       3         0
## 3     3 MALE       39      44      47         0   433     1       3         0
## 4     4 FEMALE     44      55      66         1   550     1       1         1
## 5     5 MALE       43      56      53         0   507     1       1         1
## 6     6 MALE       57      49      53         0   530     1       1         1
\end{verbatim}

\hypertarget{uxc9c0uxc6d0uxc758-uxbe48uxb3c4uxd45cuxc640-uxba74uxc81cuxc758-uxbe48uxb3c4uxd45cuxb97c-uxc791uxc131uxd558uxc2dcuxc624.}{%
\subsection{지원의 빈도표와 면제의 빈도표를
작성하시오.}\label{uxc9c0uxc6d0uxc758-uxbe48uxb3c4uxd45cuxc640-uxba74uxc81cuxc758-uxbe48uxb3c4uxd45cuxb97c-uxc791uxc131uxd558uxc2dcuxc624.}}

지원의 빈도표

\begin{Shaded}
\begin{Highlighting}[]
\NormalTok{sufreq <-}\StringTok{ }\KeywordTok{table}\NormalTok{(DF}\OperatorTok{$}\NormalTok{support)}
\KeywordTok{margin.table}\NormalTok{(sufreq)}
\end{Highlighting}
\end{Shaded}

\begin{verbatim}
## [1] 100
\end{verbatim}

비율로 변경

\begin{Shaded}
\begin{Highlighting}[]
\KeywordTok{prop.table}\NormalTok{(sufreq)}
\end{Highlighting}
\end{Shaded}

\begin{verbatim}
## 
##    1    2    3    4 
## 0.50 0.30 0.16 0.04
\end{verbatim}

면제의 빈도표

\begin{Shaded}
\begin{Highlighting}[]
\NormalTok{exfreq <-}\StringTok{ }\KeywordTok{table}\NormalTok{(DF}\OperatorTok{$}\NormalTok{exemption)}
\KeywordTok{margin.table}\NormalTok{(exfreq)}
\end{Highlighting}
\end{Shaded}

\begin{verbatim}
## [1] 100
\end{verbatim}

비율로 변경

\begin{Shaded}
\begin{Highlighting}[]
\KeywordTok{prop.table}\NormalTok{(exfreq)}
\end{Highlighting}
\end{Shaded}

\begin{verbatim}
## 
##   0   1 
## 0.2 0.8
\end{verbatim}

\hypertarget{uxc131uxbcc4uxc9c0uxc6d0-uxc131uxbcc4uxba74uxc81cuxc758-uxad50uxcc28uxd45cuxb97c-uxc791uxc131uxd558uxc2dcuxc624.}{%
\subsection{성별*지원, 성별*면제의 교차표를
작성하시오.}\label{uxc131uxbcc4uxc9c0uxc6d0-uxc131uxbcc4uxba74uxc81cuxc758-uxad50uxcc28uxd45cuxb97c-uxc791uxc131uxd558uxc2dcuxc624.}}

성별 vs 지원

\begin{Shaded}
\begin{Highlighting}[]
\NormalTok{sutbl <-}\StringTok{ }\KeywordTok{table}\NormalTok{(DF}\OperatorTok{$}\NormalTok{gender, DF}\OperatorTok{$}\NormalTok{support)}
\NormalTok{sutbl <-}\StringTok{ }\KeywordTok{xtabs}\NormalTok{(}\OperatorTok{~}\NormalTok{gender }\OperatorTok{+}\StringTok{ }\NormalTok{support, }\DataTypeTok{data=}\NormalTok{DF)}
\NormalTok{sutbl}
\end{Highlighting}
\end{Shaded}

\begin{verbatim}
##         support
## gender    1  2  3  4
##   MALE   22 12  6  1
##   FEMALE 28 18 10  3
\end{verbatim}

비율로 변경

\begin{Shaded}
\begin{Highlighting}[]
\KeywordTok{prop.table}\NormalTok{(sutbl)}
\end{Highlighting}
\end{Shaded}

\begin{verbatim}
##         support
## gender      1    2    3    4
##   MALE   0.22 0.12 0.06 0.01
##   FEMALE 0.28 0.18 0.10 0.03
\end{verbatim}

성별 vs 면제

\begin{Shaded}
\begin{Highlighting}[]
\NormalTok{extbl <-}\StringTok{ }\KeywordTok{table}\NormalTok{(DF}\OperatorTok{$}\NormalTok{gender, DF}\OperatorTok{$}\NormalTok{exemption)}
\NormalTok{extbl <-}\StringTok{ }\KeywordTok{xtabs}\NormalTok{(}\OperatorTok{~}\NormalTok{gender}\OperatorTok{+}\NormalTok{exemption, }\DataTypeTok{data=}\NormalTok{DF)}
\NormalTok{extbl}
\end{Highlighting}
\end{Shaded}

\begin{verbatim}
##         exemption
## gender    0  1
##   MALE    7 34
##   FEMALE 13 46
\end{verbatim}

비율로 변경

\begin{Shaded}
\begin{Highlighting}[]
\KeywordTok{prop.table}\NormalTok{(extbl)}
\end{Highlighting}
\end{Shaded}

\begin{verbatim}
##         exemption
## gender      0    1
##   MALE   0.07 0.34
##   FEMALE 0.13 0.46
\end{verbatim}

\end{document}
